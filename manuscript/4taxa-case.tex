\documentclass[12pt,letterpaper]{article}

\usepackage[english]{babel}

% Page Layout
\usepackage[hmargin={1.0in, 1.0in}, vmargin={1.0in, 1.0in}]{geometry}

% Spacing
\usepackage{setspace}
% Use \singlespacing, \onehalfspacing, or \doublespacing,
% or alternatively \setstretch{3} for triple spacing (or any other number).

% Mathematical Notation
\usepackage{amsmath,amstext,amssymb}

\renewcommand{\Pr}{\mathsf{P}}
\newcommand{\prob}[1]{\Pr\left(#1\right)}
\newcommand{\given}{\mid}
\newcommand{\me}{\mathrm{e}}
\renewcommand{\emptyset}{\varnothing} % use a circle instead of a zero for the empty set, requires amssymb

% our commands:
\usepackage{color}
\newcommand{\help}[1]{\textcolor{blue}{#1}}
\newcommand{\falta}[1]{\textcolor{red}{#1}}

%\usepackage{alltt}
% for mathematical notation in a verbatim-like environment
% \begin{alltt} ... \end{alltt}

% Graphics
\usepackage{graphicx}
%\usepackage[small]{subfigure}
% for subfigures in a single figure
%\usepackage{epsfig,rotating,epsf,psfrag,lscape}

% Lists
\usepackage{enumitem}
% usage: \begin{enumerate}[resume] will continue numbering from previous enumerate block

% Citations
\usepackage{natbib}


\usepackage{Sweave}
\begin{document}
\begin{center}
\textbf{Importance sampling: 4 taxa case}
\end{center}

\paragraph{Data.} 4 sequences for cats: cat, tiger, leopard, clouded
leopard in phylip file \texttt{4taxa-cats.phy}:

\texttt{Cat ATGTTCATAAACCGGTGACTATTTTCAACTAATCACAAACTGAGCTGGCATGGTGGGGACTGC...}

\texttt{CloudedLeopard ATGTTCATAAACCGCTGACTATTTTCAACTAACCATCGCTTGGGCCGGTATAGTA...}

\texttt{Leopard ATGTTCATAAACCGCTGACTATTTTCAACCAATCACAAAGATAGCTGGCATGGTGGGGACTGC...}

\texttt{Tiger ATGTTCATAAACCGCTGACTATTTTCAACCAATCACAAGGATATTTGGTATAGTGGGGACTGC...}

% fixit: need a section on model and model assumptions

\paragraph{Conditional clade distribution.} From the phylip input
file, we obtain the conditional clade distribution from a sample bootstrapped
NJ trees with the perl script \texttt{seq2ccdprobs.pl}.

\paragraph{Sample topology.} From the clade distribution, we sample
one topology. \falta{still don't know how? which output file to use?}

\paragraph{Sample branch lengths given a topology.} Let $T$ be the 4-taxon
topology sampled from the conditional clade distribution.
\begin{enumerate}
\item Choose one tip at random to exclude. Denote the other three sequences by
  $seq_1,seq_2,seq_3$, where $seq_1,seq_2$ are sisters.
\item Compute the matrices of counts between all pairs of the three
  sequences: $x_{12}, x_{13}, x_{23}$, and simulate the branch length between
  each pair with JC (or TN) model, and $\eta=0.5$ (see JCvsTN.pdf for
  details on choosing $\eta$): $d_{12},d_{13},d_{23}$
\item Compute the distances between $seq_1,seq_2$ and its parent $x$:
\begin{align*}
  d_{1x} &= \frac{(d_{12}+d_{13}-d_{23})}{2} \\
  d_{2x} &= \frac{(d_{12}+d_{23}-d_{13})}{2}
\end{align*}
\item Convert $seq_1,seq_2,seq_3,seq_4$ to matrices like
\begin{center}
\texttt{Cat ATGTTCAT...}\\[0.5cm]
$S_{cat}=$
\begin{tabular}{cccccccc}
A & 1 & 0 & 0 & 0 & 0 & 0 & 1 ...\\
C & 0 & 0 & 0 & 0 & 0 & 1 & 0 ...\\
G & 0 & 0 & 1 & 0 & 0 & 0 & 0 ...\\
T & 0 & 1 & 0 & 1 & 1 & 0 & 0 ...
\end{tabular}
\end{center}
\item Estimate the sequence distribution at $x$ from the sequences
  $seq_1,seq_2$. The formula for the likelihood at site $j$ for node
  $x$, parent of $1,2$ is:
\begin{align*}
L^x_j(s) &= \left[ \sum_{i \in \{A,C,G,T\}} P_{si}(d_{1x}) L^1_j(i)
\right] * \left[ \sum_{i \in \{A,C,G,T\}} P_{si}(d_{2x}) L^2_j(i)
\right] \\
s &\in \{A,C,G,T\}
\end{align*}
where
\begin{align*}
L^k_j(i) &= S_k[i,j], k=1,2 \\
P(t) &= \exp(\mathbf{Q}t)
\end{align*}
\falta{Q: how to estimate this Q? Also, $L^x_j$ is the likelihood, we
  want $P(x|1,2)$}\\
So that the sequence matrix for $x$ is given by $S_x$:
\begin{align*}
S_x[i,j] \falta{= L^x_j(i)}
\end{align*}
\begin{center}
$S_x=$
\begin{tabular}{cccccccc}
A & $L^x_1(A)$ & $L^x_2(A)$...\\
C & $L^x_1(C)$ & $L^x_2(C)$...\\
G & $L^x_1(G)$ & $L^x_2(G)$... \\
T & $L^x_1(T)$ & $L^x_2(T)$...
\end{tabular}
\end{center}
%aqui voy: falta d34, d3x, d4x
\item Compute the count matrix between $seq_3,seq_4$: $x_{34}$ and
  simulate the branch length $d_{34}$ with JC (or TN) model and
  $\eta=0.5$
\item Compute the equivalent to the count matrix between $x$ and
  $seq_3,seq_4$:
\begin{align*}
x_{x3}=\sum_{j=1}^{nsites} S_3[,j]*S_x[,j]^T
\end{align*}
\item Simulate branch lengths $d_{3x}, d_{4x}$ with JC (or TN) model
  and $\eta=0.5$
\item Compute the distances between $seq_3,seq_4$ and its parent $y$,
  and between $x,y$:
\begin{align*}
  d_{3y} &= \frac{(d_{34}+d_{3x}-d_{4x})}{2} \\
  d_{4y} &= \frac{(d_{34}+d_{4x}-d_{3x})}{2} \\
  d_{xy} &= \frac{(d_{3x}+d_{4x}-d_{34})}{2}
\end{align*}

\end{enumerate}

\paragraph{Compute importance weight}
\begin{align*}
w(tree) &= posterior(tree)/ g(tree) \\
g(tree) &= p(topology) density(branch lengths | topology)
\end{align*}
where $p(topology)$ comes from the clade distribution.


\end{document}