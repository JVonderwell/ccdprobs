\documentclass[12pt,letterpaper]{article}

\usepackage[english]{babel}

% Page Layout
\usepackage[hmargin={1.0in, 1.0in}, vmargin={1.0in, 1.0in}]{geometry}

% Spacing
\usepackage{setspace}
% Use \singlespacing, \onehalfspacing, or \doublespacing,
% or alternatively \setstretch{3} for triple spacing (or any other number).

% Mathematical Notation
\usepackage{amsmath,amstext,amssymb}

\renewcommand{\Pr}{\mathsf{P}}
\newcommand{\prob}[1]{\Pr\left(#1\right)}
\newcommand{\given}{\mid}
\newcommand{\me}{\mathrm{e}}
\renewcommand{\emptyset}{\varnothing} % use a circle instead of a zero for the empty set, requires amssymb

% our commands:
\newcommand{\help}[1]{\textcolor{blue}{#1}}
\newcommand{\falta}[1]{\textcolor{red}{#1}}

%\usepackage{alltt}
% for mathematical notation in a verbatim-like environment
% \begin{alltt} ... \end{alltt}

% Graphics
\usepackage{graphicx}
%\usepackage[small]{subfigure}
% for subfigures in a single figure
%\usepackage{epsfig,rotating,epsf,psfrag,lscape}

% Lists
\usepackage{enumitem}
% usage: \begin{enumerate}[resume] will continue numbering from previous enumerate block

% Citations
\usepackage{natbib}


\usepackage{Sweave}
\begin{document}
\begin{center}
\textbf{Ideas on sampling branch lengths given two sequences}
\end{center}

The requirement is to find a method that is reasonably fast
to generate at random (with a probability density that can be computed)
branch lengths for a tree topology that are not too different from an actual posterior distribution
for the purposes of implementing an importance sampling method for Bayesian phylogenetics.
The approach outlined here requires a method to generate a random distance between two ``sequences'',
where ``sequences'' is quoted because a sequence will be generalized to a site-wise independent probability distribution.

\paragraph{Jukes-Cantor model.---}
Data: X= number of sites that differ between 2 sequences; n= total
number of sites
\begin{align*}
X|p &\sim Binomial(n,p)\\
p &\sim Beta(\alpha_0,\beta_0)\\
p|X &\sim Beta(\alpha=\alpha_0+X,\beta=\beta_0+n-X)
\end{align*}

\textbf{Idea:}
\begin{enumerate}
\item Draw beta distribution
\item Rescale to $(0,3/4)$ instead of $(0,1)$
\item Plug into $\hat{t}_{MLE}$ formula:
\begin{align*}
\hat{t}_{MLE}=-\frac{3}{4} \log \left( 1- \frac{4}{3} \frac{x}{n} \right)
\end{align*}
\end{enumerate}

We introduce the parameter $\eta$ to control the variance of the JC
density for $T$:
\begin{align*}
\frac{x}{n}&=\frac{3\alpha}{a(\alpha+\beta)}\\
\alpha+\beta &= \eta n
\end{align*}
Note that the presence of $\eta$ is multiplying the beta variance by $1/ \eta$.

So, the parameters for beta are
\begin{align*}
\alpha&=\frac{4}{3}\eta x \\
\beta&=\eta n - \frac{4}{3} \eta x
\end{align*}

\textbf{Procedure:}
\begin{enumerate}
\item Simulate $W \sim Beta(\alpha,\beta)$
\item Rescale $0.75W=x/n$
\item Compute $T=-0.75 \log(1-W)$
\end{enumerate}
We could also simulate $1-W \sim Beta(\beta,\alpha)$.
%% fixit: do we want to add here density of T?

\textbf{Code:} \texttt{branch-length.r}
\begin{Schunk}
\begin{Sinput}
> Q = randomQ(n=4,rescale=TRUE)
> x = simulateSequenceSummary(nsites=500,Q=Q,s=0.15)
> w = simulateBranchLength.jc(nsim=100,x=x,eta=0.8)
\end{Sinput}
\end{Schunk}


\paragraph{Tamura-Nei model.---}
add assumptions

\begin{align*}
  % n &= sum(x) \\
  % prop.ag &= (x[1,3] + x[3,1]) / n \\
  % prop.ct &= (x[2,4] + x[4,2]) / n \\
  % prop.tv &= (x[1,2] + x[1,4] + x[2,1] + x[2,3] + x[3,2] + x[3,4] +
  % x[4,1] &+ x[4,3]) / n \\
  % p.est &= (apply(x,1,sum) + apply(x,2,sum)) / (2*n) \\
  % p.a &= p.est[1] \\
  % p.c &= p.est[2] \\
  % p.g &= p.est[3] \\
  % p.t &= p.est[4] \\
  % p.r &= sum(p.est[c(1,3)]) \\
  % p.y &= sum(p.est[c(2,4)]) \\
  numer1 &= 2*p.a*p.g*p.r \\
  denom1 &= numer1 - p.r^2*prop.ag - p.a*p.g*prop.tv \\
  c1 &= numer1 / denom1 \\
  numer2 &= 2*p.c*p.t*p.y \\
  denom2 &= numer2 - p.y^2*prop.ct - p.c*p.t*prop.tv \\
  c2 &= numer2 / denom2 \\
  c3 &= (2*p.a^2*p.g^2) / (p.r * denom1) + (2*p.c^2*p.t^2) / (p.y * denom2) \\
  &+ (p.r^2 * (p.c^2 + p.t^2) + p.y^2 * (p.a^2 + p.g^2) ) /(2*p.r^2*p.y^2 - p.r*p.y*prop.tv) \\
  mu &= -2 * ( (p.a*p.g/p.r) * log(1 - p.r*prop.ag/(2*p.a*p.g) - prop.tv/(2*p.r) )\\
  &+ (p.c*p.t/p.y) * log(1 - p.y*prop.ct/(2*p.c*p.t) - prop.tv/(2*p.y) ) \\
  &+
  (p.r*p.y - p.a*p.g*p.y/p.r - p.c*p.t*p.r/p.y) * log(1 -
  prop.tv/(2*p.r*p.y)) ) \\
  v &= (1/ eta) * ((c1^2*prop.ag + c2^2*prop.ct + c3^2*prop.tv) \\
&-(c1*prop.ag + c2*prop.ct + c3*prop.tv)^2)/n
\end{align*}
Note that the presence of $\eta$ is multiplying the beta variance by $1/ \eta$.

\textbf{Procedure:}
\begin{enumerate}
\item Simulate $W \sim Gamma(\mu^2/v,\mu/v)$
\end{enumerate}

\textbf{Code:} \texttt{branch-length.r}
\begin{Schunk}
\begin{Sinput}
> Q = randomQ(n=4,rescale=TRUE)
> x = simulateSequenceSummary(nsites=500,Q=Q,s=0.15)
> w = simulateBranchLength.tn(nsim,x,eta=0.9)
\end{Sinput}
\end{Schunk}


\section{Comparison JC vs TN}
\subsection{Bias}
\textbf{Code:} \texttt{JCvsTN.r}

\begin{Schunk}
\begin{Sinput}
> s=0.15
> nrep=1000
> nsites=1000
> eta=0.8
\end{Sinput}
\end{Schunk}


\begin{itemize}
\item bias=true mode - JC(TN) mode
\item Most times bias positive ($\sim 0.926$)
\item Most times ($\sim 0.99$), bias less than $0.02$ (spread density $\sim 0.1$).
\item JC bias is systematically bigger than TN bias (see figures).
\item When one of them is surprisingly big, the other is too
\item Big biases associated to too negative diagonal values in Q ($<-5.0$)
\end{itemize}

\textbf{Problems with TN}
\begin{itemize}
\item Errors with rows (or columns) of zero in the count matrix
\item Errors if row (or column) only one entry different than zero (or
  close to vector of zeros): $NaN$ caused by negative argument in the logarithm.
\item This is also associated with very small values in the diagonal
  of Q matrix, but not as small as causing big bias ($<-2.2$).
\end{itemize}

\begin{figure}
\centering
\includegraphics[scale=0.5]{biasTrue1000_2.pdf}
\end{figure}

\begin{figure}
\centering
\includegraphics[scale=0.5]{biasGTR1000_2.pdf}
\end{figure}

\begin{figure}
\centering
\includegraphics[scale=0.5]{JCvsTN1000_2.pdf}
\end{figure}

\pagebreak

\subsection{Coverage}

\textbf{Code:} \texttt{JCvsTN.r}

\begin{figure}[ht]
\centering
\includegraphics[scale=0.5]{cov.pdf}
\caption{Percentage of coverage with the cutoff method}
\end{figure}

\begin{figure}[ht]
\centering
\includegraphics[scale=0.5]{covCI.pdf}
\caption{Percentage of coverage with the 95\% CI}
\end{figure}

\pagebreak

\subsubsection{Densities}

\begin{figure}[ht]
\centering
\includegraphics[width=7cm, height=7cm]{densities_eta02.pdf}
\includegraphics[width=7cm, height=7cm]{densities_eta06.pdf}
\includegraphics[width=7cm, height=7cm]{densities_eta08.pdf}
\caption{$\eta=0.2,0.6,0.8$}
\end{figure}


\end{document}





